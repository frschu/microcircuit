\documentclass[portrait,final, fontscale=.34]{baposter}
%\documentclass[a4shrink,portrait,final]{baposter}
% Usa a4shrink for an a4 sized paper.

\tracingstats=2

\usepackage{times}
\usepackage[utf8]{inputenc}
\usepackage{calc}
\usepackage{graphicx}
\usepackage{amsmath}
\usepackage{amssymb}
\usepackage{relsize}
\usepackage{multirow}
\usepackage{bm}
\usepackage[all]{xy}

\usepackage{graphicx}
\usepackage{multicol}
\usepackage{latexsym}         % dito
\usepackage{eucal}            % nicer caligraphic fonts
\usepackage{bbm}              % nicer bbm fonts
\usepackage{mathrsfs}         % nette kaligraphische Schrift

\usepackage{pgfbaselayers}
\pgfdeclarelayer{background}
\pgfdeclarelayer{foreground}
\pgfsetlayers{background,main,foreground}

\usepackage{helvet}
%\usepackage{bookman}
\usepackage{palatino}

\newcommand{\captionfont}{\footnotesize}

\selectcolormodel{cmyk}

\graphicspath{{images/}}

%%%%%%%%%%%%%%%%%%%%%%%%%%%%%%%%%%%%%%%%%%%%%%%%%%%%%%%%%%%%%%%%%%%%%%%%%%%%%%%%
%%%% Some math symbols used in the text
%%%%%%%%%%%%%%%%%%%%%%%%%%%%%%%%%%%%%%%%%%%%%%%%%%%%%%%%%%%%%%%%%%%%%%%%%%%%%%%%

\newcommand{\x}{\ensuremath{\textbf{x}^\ast}}
\newcommand{\xc}{\textbf{{x}}}
\newcommand{\y}{\textbf{y}}
\newcommand{\vu}{\textbf{u}}
\newcommand{\p}{\ensuremath{\boldsymbol{\theta}}}
\newcommand{\X}{\textbf{X}}
\newcommand{\parder}[2]{\frac{\partial #1}{\partial #2}}



%%%%%%%%%%%%%%%%%%%%%%%%%%%%%%%%%%%%%%%%%%%%%%%%%%%%%%%%%%%%%%%%%%%%%%%%%%%%%%%%
% Multicol Settings
%%%%%%%%%%%%%%%%%%%%%%%%%%%%%%%%%%%%%%%%%%%%%%%%%%%%%%%%%%%%%%%%%%%%%%%%%%%%%%%%
\setlength{\columnsep}{1em}
\setlength{\columnseprule}{0mm}


%%%%%%%%%%%%%%%%%%%%%%%%%%%%%%%%%%%%%%%%%%%%%%%%%%%%%%%%%%%%%%%%%%%%%%%%%%%%%%%%
% Packages
%-------------------------------------------------------------------------------
% The only ``weird'' dependency of this package is pgf. All the rest should be
% installed on any decent system.
%%%%%%%%%%%%%%%%%%%%%%%%%%%%%%%%%%%%%%%%%%%%%%%%%%%%%%%%%%%%%%%%%%%%%%%%%%%%%%%%
\typeout{Use Packages}
\usepackage{xkeyval}
\usepackage{calc}
\usepackage[cmyk]{xcolor}
\usepackage{tikz}
\usepackage{pgf}
\usepackage{ifthen}
\usepackage[T1]{fontenc}

\usetikzlibrary{decorations}

%%%%%%%%%%%%%%%%%%%%%%%%%%%%%%%%%%%%%%%%%%%%%%%%%%%%%%%%%%%%%%%%%%%%%%%%%%%%%%%%
% Settings
%%%%%%%%%%%%%%%%%%%%%%%%%%%%%%%%%%%%%%%%%%%%%%%%%%%%%%%%%%%%%%%%%%%%%%%%%%%%%%%%
% TODO: Add package options
\typeout{Setup}

% Choose a smaller value for larger fonts
\newcommand{\baposter@fontscale}{0.292}
% Landscape versus portrait
\newcommand{\baposter@format}{}
% default paper size
\newcommand{\baposter@papersize}{a0paper}
% use the ``showframe'' option of the ``geometry'' package?
\newcommand{\baposter@showframe}{false}


%%%%%%%%%%%%%%%%%%%%%%%%%%%%%%%%%%%%%%%%%%%%%%%%%%%%%%%%%%%%%%%%%%%%%%%%%%%%%%%%
% Save space in lists. Use this after the opening of the list
%%%%%%%%%%%%%%%%%%%%%%%%%%%%%%%%%%%%%%%%%%%%%%%%%%%%%%%%%%%%%%%%%%%%%%%%%%%%%%%%
\newcommand{\compresslist}{%
\setlength{\itemsep}{1pt}%
\setlength{\parskip}{0pt}%
\setlength{\parsep}{0pt}%
}


%%%%%%%%%%%%%%%%%%%%%%%%%%%%%%%%%%%%%%%%%%%%%%%%%%%%%%%%%%%%%%%%%%%%%%%%%%%%%%
%%% Begin of Document
%%%%%%%%%%%%%%%%%%%%%%%%%%%%%%%%%%%%%%%%%%%%%%%%%%%%%%%%%%%%%%%%%%%%%%%%%%%%%%

\begin{document}

%%%%%%%%%%%%%%%%%%%%%%%%%%%%%%%%%%%%%%%%%%%%%%%%%%%%%%%%%%%%%%%%%%%%%%%%%%%%%%
%%% Here starts the poster
%%%---------------------------------------------------------------------------
%%% Format it to your taste with the options
%%%%%%%%%%%%%%%%%%%%%%%%%%%%%%%%%%%%%%%%%%%%%%%%%%%%%%%%%%%%%%%%%%%%%%%%%%%%%%
% Define some colors
\definecolor{silver}{cmyk}{0,0,0,0.3}
\definecolor{texthintergrund}{RGB}{88,125,199}
\definecolor{rahmen}{RGB}{34,80,169}
\definecolor{black}{cmyk}{0,0,0.0,1.0}
\definecolor{darkYellow}{cmyk}{0,0,1.0,0.5}
\definecolor{darkSilver}{cmyk}{0,0,0,0.1}

\definecolor{boxueberschrift}{RGB}{230,217,202}
%\definecolor{hintergrundDunkel}{RGB}{197,182,165}
\definecolor{hintergrundDunkel}{RGB}{88,125,199}

\definecolor{hintergrundHell}{RGB}{255, 255,255}


\definecolor{uniblau}{cmyk}{1.0,0.48,0,0.35}
\definecolor{uniblau2}{HTML}{004A90}

%%
\typeout{Poster Starts}
\background{
  \begin{tikzpicture}[remember picture,overlay]%
    \draw (current page.north west)+(-2em,2em) node[anchor=north west] {\includegraphics[height=1.05\textheight]{drawing.pdf}};
  \end{tikzpicture}%
}

\newlength{\leftimgwidth}
\begin{poster}%
  % Poster Options
  {
  % Show grid to help with alignment
  grid=no,
  % Column spacing
  colspacing=1em,
  % Color style
  bgColorOne=hintergrundDunkel,
  bgColorTwo=hintergrundHell,
  borderColor=rahmen,
  headerColorOne=texthintergrund,
  headerColorTwo=rahmen,
  headerFontColor=white,
  boxColorOne=white,
  boxColorTwo=hintergrundDunkel,
  % Format of textbox
  textborder=rectangle,
  % Format of text header
  eyecatcher=no,
  headerborder=open,
  headerheight=0.1\textheight,
  headershape=rectangle,
  headershade=plain,
  headerfont=\Large\textsf, %Sans Serif
  boxshade=plain,
  %background=shade-tb,
  background=user,
  linewidth=2pt
  }
  % Eye Catcher
  {\includegraphics[width=10em]{D1077}} % No eye catcher for this poster. (eyecatcher=no above). If an eye catcher is present, the title is centered between eye-catcher and logo.
  % Title
  {\sf %Sans Serif
  %\bf% Serif
An application of Lie group theory to detect structural non-identifiabilities in biological pathway models
\vspace{0em}}
  % Authors
  {\sf %Sans Serif
  % Serif
  \vspace{0em}\Large{Benjamin Merkt$^{1}$, Daniel Kaschek$^1$, Jens Timmer$^{1,2,3}$\\
  \normalsize
  $^1$Institute of Physics, Freiburg University -- $^2$ZBSA, Freiburg Center for Systems Biology -- $^3$BIOSS, Centre for Biological Signalling Studies, Freiburg 
  }
  }
  % University logo
  {% The makebox allows the title to flow into the logo, this is a hack because of the L shaped logo.
    \makebox[8em][r]{%
      \begin{minipage}{16em}
        \hfill
        \includegraphics[height=10em]{unisiegel}
      \end{minipage}
    }
  }


%%%%%%%%%%%%%%%%%%%%%%%%%%%%%%%%%%%%%%%%%%%%%%%%%%%%%%%%%%%%%%%%%%%%%%%%%%%%%%
%%% Now define the boxes that make up the poster
%%%---------------------------------------------------------------------------
%%% Each box has a name and can be placed absolutely or relatively.
%%% The only inconvenience is that you can only specify a relative position 
%%% towards an already declared box. So if you have a box attached to the 
%%% bottom, one to the top and a third one which should be in between, you 
%%% have to specify the top and bottom boxes before you specify the middle 
%%% box.
%%%%%%%%%%%%%%%%%%%%%%%%%%%%%%%%%%%%%%%%%%%%%%%%%%%%%%%%%%%%%%%%%%%%%%%%%%%%%%


%\begin{footnotesize}
%%%%%%%%%%%%%%%%%%%%%%%%%%%%%%%%%%%%%%%%%%%%%%%%%%%%%%%%%%%%%%%%%%%%%%%%%%%%%%
  \headerbox{Abstract}{name=abstract,column=0,row=0, span=3}{
%%%%%%%%%%%%%%%%%%%%%%%%%%%%%%%%%%%%%%%%%%%%%%%%%%%%%%%%%%%%%%%%%%%%%%%%%%%%%%
   
\begin{multicols}{3}
\begin{footnotesize}
In Systems Biology, chemical reaction networks with a high number of species and a low number of observed quantities are very common. In addition, many measurement techniques like quantitative Western blot or protein arrays provide only relative data leading to undetermined scaling parameters in the mathematical model. This results in the problem of structural non-identifiability:  the effect of changing one parameter can be compensated by other parameters.

To find non-identifiabilities in a system of ordinary differential equations (ODEs), Lie group theory can be utilized. Being designed for searching symmetries in almost all kinds of algebraic and differential equations, the theory leads to groups of transformations of the internal variables and parameters leaving the observation invariant. Thus, Lie symmetries are one source for structural non-identifiability.

Based on the results of Lie theory, model non-identifiabilities can be effectively detected and resolved by Lie group transformations. For the most common transformations like scaling transformations and translations, the analysis can be performed within a few seconds for systems containing dozens of variables and parameters. In the future we hope to be able to extend this method to find more complicated transformations.
\end{footnotesize}
\end{multicols}
\vspace{-0.5em}
}

%%%%%%%%%%%%%%%%%%%%%%%%%%%%%%%%%%%%%%%%%%%%%%%%%%%%%%%%%%%%%%%%%%%%%%%%%%%%%%
  \headerbox{Motivating Example}{name=example,column=0,span=1.5, below=abstract}{
%%%%%%%%%%%%%%%%%%%%%%%%%%%%%%%%%%%%%%%%%%%%%%%%%%%%%%%%%%%%%%%%%%%%%%%%%%%%%%
Consider the example $\begin{xy}\xymatrix{**[l]A+A \ar@<1.5pt>[r]^{k_1}    &   B\ar@<1.5pt>[l]^{k_2}}	\end{xy}$:
%Consider the following example:
%\begin{align*}
%\begin{xy}
%	\xymatrix{**[l]A+A \ar@<1.5pt>[r]^{k_1}    &   B\ar@<1.5pt>[l]^{k_2}}
%\end{xy}
%\end{align*}

\vspace{-10pt}
\begin{align*}
	\dot{A}(t) &= -2\;k_1\;A^2(t) + 2\;k_2\;B(t) & A_{obs} &= s\;A(t).\\
	\dot{B}(t) &= +\;k_1\;A^2(t) - k_2\;B(t)	
\end{align*}

It admits the transformation
\begin{align*}
	\tilde{A}(t) &= e^{\epsilon} \: A(t) & \tilde{k}_1 &= e^{-\epsilon}\: k_1\\
	\tilde{B}(t) &= e^{\epsilon} \: B(t) & \tilde{s} &= e^{-\epsilon}\: s.
\end{align*}

\begin{center}{\includegraphics[width=.9\textwidth, clip = true, trim = 0pt 0pt 0pt 0pt]{images/bsp.pdf}}\end{center}

    }

%%%%%%%%%%%%%%%%%%%%%%%%%%%%%%%%%%%%%%%%%%%%%%%%%%%%%%%%%%%%%%%%%%%%%%%%%%%%%%
  \headerbox{Theory of Lie Groups}{name=theory,column=0, span=1.5,below=example}{
%%%%%%%%%%%%%%%%%%%%%%%%%%%%%%%%%%%%%%%%%%%%%%%%%%%%%%%%%%%%%%%%%%%%%%%%%%%%%%
In general, an ODE model has the form
\begin{align*}
	\dot{\xc} &= f (\xc,\p,\vu)  & \y &= g(\xc,\p).
\end{align*}
The goal is to find transformations $\tilde{\textbf{x}}^\ast = \Psi(\x, \epsilon)$ in the extended state space \mbox{$\x = \{\xc,\p,\vu\}$} which leave the observation \y{} invariant. Every Lie transformation can be written as
%\begin{align*}
%	\tilde{\x} &= e^{\epsilon \X}\:\x & \X &= \sum_i \eta^i(\x) \frac{\partial}{\partial x_i} & \eta^i &= \frac{\partial \Psi}{\partial \epsilon}\bigg|_{\epsilon=0}.
%\end{align*}
\begin{align*}
	\tilde{\textbf{x}}^\ast  = \Psi(\x, \epsilon)= e^{\epsilon \X}\:\x 
\end{align*}
with the infinitesimal generator
\begin{align*}
	\X &= \sum_i \eta^i(\x) \frac{\partial}{\partial x'_i} & \eta^i(\x) &= \frac{\partial \Psi}{\partial \epsilon}\bigg|_{\epsilon=0}.
\end{align*}
The system admits this transformation if and only if the determining equations
%\begin{align}
%	\begin{aligned}
%	\X' \cdot(f_k(\x)-\dot{x}^\ast_k) = \sum_{i=1}^n \eta_i(\x)\; \parder{f_k}{x^\ast_i}(\x)  - \sum_{j=1}^n \parder{\eta_k}{x^\ast_j}(\x) \;f_j(\x) &=0 & k &= 1,\dots,n\\
%	\X\cdot(g_{k}(\x)-y_{k}) = \sum_{i=1}^n \eta_i(\x)\; \parder{g_{k}}{x^\ast_i}(\x) &=0 & k &= 1,\dots,m
%	\end{aligned}\label{eq:deteq}
%\end{align}
\begin{align}
	\begin{aligned}
	\X' \cdot(f_k-\dot{x}^\ast_k) &= \sum_{i=1}^n \eta_i\; \parder{f_k}{x^\ast_i}  - \sum_{j=1}^n \parder{\eta_k}{x^\ast_j} \;f_j =0 & k &= 1,\dots,n\\
	\X\cdot(g_{k}-y_{k}) &= \sum_{i=1}^n \eta_i\; \parder{g_{k}}{x^\ast_i} =0 & k &= 1,\dots,m
	\end{aligned}\label{eq:deteq}
\end{align}
hold.
 }


%%%%%%%%%%%%%%%%%%%%%%%%%%%%%%%%%%%%%%%%%%%%%%%%%%%%%%%%%%%%%%%%%%%%%%%%%%%%%%
  \headerbox{Polynomial Generators}{name=polynomial,column=0,span=1.5, below=theory}{
%%%%%%%%%%%%%%%%%%%%%%%%%%%%%%%%%%%%%%%%%%%%%%%%%%%%%%%%%%%%%%%%%%%%%%%%%%%%%%
Some simple infinitesimals lead to common transformations:
\begin{align*}
	 \eta^i &= r_i &\Leftrightarrow \tilde x_i^\ast &= x_i^\ast + r_i & \text{(translation),}\\
	\eta^i &= r_i \, x_i^\ast &\Leftrightarrow \tilde x_i^\ast &= e^{r_i} x_i^\ast & \text{(scaling),} \\
	 \eta^i &= r_i \, {x_i^\ast}^2  &\Leftrightarrow \tilde x_i^\ast &= \frac{x_i^\ast}{1-r_i x_i^\ast}  & \text{(Michaelis-Menten).}
\end{align*}
Therefore, the ansatz
\begin{align*}
	\eta^i = \sum_p r_{i,p} \;{x^\ast_i}^p
\end{align*}
catches the most prominent transformations. For rational ODEs, equations \eqref{eq:deteq} can be transformed to polynomials with coefficient functions $h^{k}_{\vec{i}}$ which are linear in the $r_{i,p}$s:
\begin{align*}
	\sum_{\vec{j}} h^{k}_{\vec{j}}(\text{r}_1,\dots,\text{r}_n)\cdot {x^\ast_1}^{j_1}\cdots {x^\ast_n}^{j_n} &= 0 & &\forall k.
%\end{align*}
%\begin{align*}
	%\Leftrightarrow g^k_j(r_1,\dots,r_n) = 0 \quad\text{for all}\quad k
\end{align*}
This can be efficiently solved by a computer algebra system for systems even with dozens of variables.
 }



\newcommand{\nfkb}[0]{\text{NF}\ensuremath{\kappa}\text{B}}
\newcommand{\nfKb}[0]{\text{NF}\ensuremath{\boldsymbol\kappa}\text{B}}
\newcommand{\ikb}{\text{I}\ensuremath{\kappa}\text{B} }
\newcommand{\ikba}{\text{I}\ensuremath{\kappa}\text{B}\ensuremath{\alpha}}


%%%%%%%%%%%%%%%%%%%%%%%%%%%%%%%%%%%%%%%%%%%%%%%%%%%%%%%%%%%%%%%%%%%%%%%%%%%%%%
  \headerbox{ODE Model of the NF$\kappa$B Pathway}{name=model,column=1.5,span=1.5, below=abstract}{
%%%%%%%%%%%%%%%%%%%%%%%%%%%%%%%%%%%%%%%%%%%%%%%%%%%%%%%%%%%%%%%%%%%%%%%%%%%%%%

\textbf{Model:}
\begin{center}\includegraphics[scale=.48,clip=true,trim=30pt 220pt 70pt 30pt]{images/BMD1.pdf}\end{center}

%\begin{align}
%	\partial_t \nfkb{}\!\circ\!\ikba{} &= + k_{10}\cdot\text{n}\nfkb{}\!\circ\!\text{n}\ikba{} - \left(\frac{k_1\cdot\text{pIKK}}{1+k_{11}\cdot \text{pIKK}} + k_1'\right)  \cdot\nfkb{}\!\circ\!\ikba{} \notag\\
%	\partial_t \text{p}\nfkb{}\!\circ\!\text{p}\ikba{} &= + \left(\frac{k_1\cdot\text{pIKK}}{1+k_{11}\cdot \text{pIKK}} + k_1'\right)  \cdot\nfkb{}\!\circ\!\ikba{} - k_2\cdot\text{p}\nfkb{}\!\circ\!\text{p}\ikba{}\notag\\
%	\partial_t \text{p}\nfkb{} &= +k_2\cdot\text{p}\nfkb{}\!\circ\!\text{p}\ikba{} - k_3\cdot\text{p}\nfkb{}\notag\\
%	\partial_t \text{p}\ikba{} &= +k_2\cdot\text{p}\nfkb{}\!\circ\!\text{p}\ikba{} - k_4\cdot\text{p}\ikba{}\notag\\
%	\partial_t \text{n}\nfkb{} &= + k_3\cdot\rho_{vol}\cdot\text{p}\nfkb{} - k_9\cdot\text{n}\ikba{}\cdot\text{n}\nfkb{}\notag\\
%	\partial_t \text{m}\ikba{} &= + k_5 \cdot\text{n}\nfkb{} - k_6\;\text{m}\ikba{}\notag\\
%	\partial_t\ikba{} &=+k_7\cdot\text{m}\ikba{} - k_8\cdot\ikba{}\notag\\
%	\partial_t\text{n}\ikba{} &= + k_8\cdot\rho_{vol}\cdot\ikba{} - k_9\cdot\text{n}\ikba{}\cdot\text{n}\nfkb{}\notag\\
%	\partial_t\text{n}\nfkb{}\!\circ\!\text{n}\ikba{} &= +k_9\cdot\text{n}\ikba{}\cdot\text{n}\nfkb{} - k_{10}\cdot\rho_{vol}\cdot\text{n}\nfkb{}\!\circ\!\text{n}\ikba{}.\notag\\\label{eq:nfkb0}
%\end{align}
%\vspace{-15pt}
\textbf{Observables:}
\begin{align*}
		\nfkb{}_\text{Cyt} &= s_1\cdot(\nfkb{}\!\circ\!\ikba{} + \text{p}\nfkb{}\!\circ\!\text{p}\ikba{} + \text{p}\nfkb{}) + I0_\text{Cyt}\\	
		\nfkb{}_\text{Nuc} &= s_2\cdot(\text{n}\nfkb{}\!\circ\!\text{n}\ikba{} + \text{n}\nfkb{}) + I0_\text{Nuc}\\
		\text{p}\nfkb{}_\text{Cyt} &= s_3\cdot(\text{p}\nfkb{}\!\circ\!\text{p}\ikba{} + \text{p}\nfkb{})\\
		\text{p}\ikba{}_\text{Cyt} &= s_4\cdot(\text{p}\nfkb{}\!\circ\!\text{p}\ikba{} + \text{p}\ikba{})
\end{align*}
}


\def\lengtho{-20pt}
\def\lengtht{0pt}
%%%%%%%%%%%%%%%%%%%%%%%%%%%%%%%%%%%%%%%%%%%%%%%%%%%%%%%%%%%%%%%%%%%%%%%%%%%%%%
  \headerbox{Symmetries of the NF$\kappa$B Model}{name=symmetries,column=1.5,span=1.5, below=model}{
%%%%%%%%%%%%%%%%%%%%%%%%%%%%%%%%%%%%%%%%%%%%%%%%%%%%%%%%%%%%%%%%%%%%%%%%%%%%%%
\textbf{Scaling symmetries:}
\begin{itemize}
\item All state variables and scaling parameters:\vspace{\lengtht}
\begin{align*}
	x_i^\ast = e^\epsilon\; x_i\quad i=1,...,9 \qquad s_j^\ast = e^{-\epsilon} s_j\quad j=1,...,4 \qquad k_9^\ast = e^{-\epsilon}\;k_9
\end{align*}
\vspace{\lengtho}\item Nuclear state variables and scaling parameters:\vspace{\lengtht}
\begin{align*}
	\text{n}\nfkb{}^\ast &= e^\epsilon\;\text{n}\nfkb{} & \rho_{vol}^\ast &= e^\epsilon\;\rho_{vol}\\
	\text{n}\ikba{}^\ast &= e^\epsilon\;\text{n}\ikba{} & k_i^\ast &= e^{-\epsilon}\;k_i\quad i=5,9,10\\
	\text{n}\nfkb{}\!\circ\!\text{n}\ikba{}^\ast &= e^\epsilon\;\text{n}\nfkb{}\!\circ\!\text{n}\ikba{} & s_2^\ast &= e^{-\epsilon}\;s_2
\end{align*}
\vspace{\lengtho}\item mRNA level:\vspace{\lengtht}
\begin{align*}
	\text{m}\ikba{}^\ast &= e^\epsilon\;\text{m}\ikba{} & k_5^\ast &= e^\epsilon\;k_5 & k_7 &= e^{-\epsilon} \;k_7.
\end{align*}
\vspace{\lengtho}\item pIKK input level:\vspace{\lengtht}
\begin{align*}
	\text{pIKK}^\ast &= e^{\epsilon}\;\text{pIKK} & k_1^\ast &= e^{-\epsilon}\;k_1 & k_{11}^\ast &= e^{-\epsilon}\;k_{11}
\end{align*}

\end{itemize}

\textbf{Higher order symmetry:}
\begin{align*}
	\text{pIKK}^\ast &= \frac{\text{pIKK}}{1-\epsilon\;\text{pIKK}} & k_{11}^\ast &= k_{11} + \epsilon.
\end{align*}

\begin{center}
\begin{minipage}{0.32\textwidth}
\center\includegraphics[width=\textwidth]{images/pIkk.pdf}
\end{minipage}
\begin{minipage}{0.32\textwidth}
\center\includegraphics[width=\textwidth]{images/pIkk_trans.pdf}
\end{minipage}
\begin{minipage}{0.32\textwidth}
\center\includegraphics[width=\textwidth]{images/pIkk_modified.pdf}
\end{minipage}
\end{center}
}

%%%%%%%%%%%%%%%%%%%%%%%%%%%%%%%%%%%%%%%%%%%%%%%%%%%%%%%%%%%%%%%%%%%%%%%%%%%%%%
  \headerbox{Summary}{name=summary,column=1.5,span=1.5, below=symmetries}{
%%%%%%%%%%%%%%%%%%%%%%%%%%%%%%%%%%%%%%%%%%%%%%%%%%%%%%%%%%%%%%%%%%%%%%%%%%%%%%

\begin{itemize}
\item Lie theory is the mathematical framework to handle transformations in differential equations which leave the observation invariant.
\item Every admitted transformation corresponds to a structural non-identifiability.
\item Using the presented infinitesimal generator approach, the number of parameters can be reduced or new experiments can be suggested such that the model eventually becomes identifiable.
\end{itemize}


}



%
%%%%%%%%%%%%%%%%%%%%%%%%%%%%%%%%%%%%%%%%%%%%%%%%%%%%%%%%%%%%%%%%%%%%%%%%%%%%%%%
%  \headerbox{References}{name=polynomial,column=1.5,span=1.5, below=summary}{
%%%%%%%%%%%%%%%%%%%%%%%%%%%%%%%%%%%%%%%%%%%%%%%%%%%%%%%%%%%%%%%%%%%%%%%%%%%%%%%
%\begin{footnotesize}
%\begin{itemize}
%\item P. J. Olver. Applications of Lie Groups to Differential Equations, \textit{Springer} (1986)
%\item G. W. Bluman, S. C. Anco. Symmetry and Integration Methods for Differnetial Equations, \textit{Springer} (2002)
%\end{itemize}
%\end{footnotesize}
%}



\hspace{20pt}\includegraphics[width=.07\textwidth]{BMBF}
\hspace{570pt}\includegraphics[width=.07\textwidth]{zbsa}
\hspace{20pt}\includegraphics[width=.07\textwidth]{bioss}


%\end{footnotesize}
\end{poster}%
\end{document}
