%*******************************************************
% Abstract
%*******************************************************
%\renewcommand{\abstractname}{Abstract}
\pdfbookmark[1]{Abstract}{Abstract}
\begingroup
\let\clearpage\relax
\let\cleardoublepage\relax
\let\cleardoublepage\relax

\section*{Abstract}
In the search for understanding the basic functions of the neocortex, 
linking the anatomical structure to population dynamics has been a mayor
focus of research. Experimental data regarding local connectivity
and cell-type specific activity is increasing at a fast pace but remains
mainly inconclusive. Simultaneously, spiking network models based on
leaky integrate-and-fire neurons 
are used for interpreting experimental data such as firing rates, 
correlations or oscillations. 
On the other hand, a deeper understanding is reached by 
an analytical framework relying on a statistical description.
A rate based mean field theory has been developed 
for neural networks of two populations
by \citeb{brunel2000} and successfully applied to a number 
spiking network models. However, for networks with many populations
it is not a priori clear whether this description is adequate. 
One important case considered in this work is the modeling of the neocortical networks. 
These networks are organized in layers, a feature thought to be 
crucial for the information processing 
and thus for higher cognitive functions.
A spiking network model
of a substructure, termed as the local cortical microcircuit, and
incorporating this characteristic
was established by \citeb{potjans2014} in 2014. 
To this end, they integrated a large number of the experimental studies 
available and reproduced some of the main features of spiking activity 
observed \textit{in vivo}.
In this thesis, the spiking network model of the microcircuit is successfully reimplemented, 
compared with the original one and further analyzed.
Aiming for a more thorough understanding,
the existing mean field theory is extended to a layered network of eight populations. 
The predictions, namely single neuron firing rates, 
membrane potential distributions as well as the irregularity of spike trains, 
are are shown to agree with the simulated counterparts.
Finally, the developed mean field theory is shown to be a convenient tool
exploring the network's activity for changing parameters. 
%\emph{Finally, in order to test 
%the dependence of a number of parameters, a simulation adapted to the 
%mean field theory is contrasted with the previous results.}

\vfill

\pdfbookmark[1]{Zusammenfassung}{Zusammenfassung}
\pagebreak
\begin{otherlanguage}{ngerman}
\section*{Zusammenfassung}

Bei der Erforschung des Neokortex ist die Verbindung zwischen Struktur und 
Funktion der zentrale Ansatz der Computational Neuroscience. 
Trotz wachsender Zahl an experimentellen Ergebnissen zu 
Konnektivitäten und zellspezifischer Aktivität gibt es noch kein zusammenhängendes 
Bild darüber, wie die wichtigsten Funktionen zustande kommen. 
Gepulste neuronale Netze (SNN -- spiking neural networks) mit leaky integrate-and-fire
Neuronen liefern einen wichtigen Ansatz, um die experimentellen Daten, 
wie etwa Feuerraten, Korrelationen oder Oszillationen, zu verstehen. Gleichzeitig wird 
versucht, ein umfassenderes Verständnis mit Hilfe von statistischen Modellen 
zu erlangen. Von \citeb{brunel2000} wurde eine ratenbasierte Molekularfeldnäherung für
Netwerke aus zwei Populationen entwickelt und bis dato auf einige SNNs angewandt. 
Es ist jedoch nicht klar, in wie weit diese Beschreibung auf komplexere Netzwerke
verallgemeinerbar ist. Ein wichtiger Fall, der in dieser Arbeit betrachtet wird, 
sind Modelle neokortikaler Netzwerke. Diese sind in Schichten aufgebaut -- 
ein Eigenschaft, die als maßgeblich für die Informationsverarbeitung und damit für 
höhere kognitive Fähigkeiten gilt, welche in diesem Teil des Gehirns zustandekommen. 
Ein SNN für den lokalen kortikalen Schaltkreis, eine 
Unterstruktur des Neocortex, wurde von \citeb{potjans2014} entwickelt. Dieses
enthält die laminare Struktur. Für die Konnektivitätsmatrix des Modells 
wurde eine Vielzahl experimenteller Studien ausgewertet, und schließlich wurden
einige wichtige Merkmale von \textit{in vivo} gemessenen Daten reproduziert. 
In dieser Arbeit wird das Modell neu implementiert, mit dem Original verglichen 
und weitergehen untersucht. 
Um das Modell besser zu verstehen wird die 
Molekularfeldtheorie auf ein geschichtetes Netzwerk aus acht Populationen erweitert. 
Beim Vergleich von vorgesagten und gemessenen Daten wird gezeigt, dass die Theorie
gut mit der Simulation übereinstimmt. Speziell werden dafür die Feuerraten, die Membranpotentiale
und die Unregelmäßigkeit der Neuronenaktivität untersucht. Schließlich wird 
gezeigt, dass sich die Molekularfeldtheorie auch als nützliches Werkzeug zur weiteren Untersuchung 
bei veränderten Netzwerkparametern eignet. 

\end{otherlanguage}






\endgroup			

\vfill
