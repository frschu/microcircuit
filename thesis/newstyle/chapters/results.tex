\chapter{Results}
\label{sec:results}

\section{Spiking network model}
The simulation of the spiking network model will be analyzed in the following manner. 
At first, the results are directly compared to those obtained by the original 
model of Potjans and Diesmann. Both simulations are run for the same parameters, 
differences arise solely due to internal differences in assigning the random 
number generators. It is therefor not feasible to do a direct (spike per spike) 
comparison. Instead, a statistical description is chosen which furthermore sheds light
onto the differences between individual realizations with different seeds for the 
random number generators.
The distribution of single neuron firing rate within the population
and the interspike interval distributions.
are then analyzed more thoroughly. 

A comparison between the PyNEST implementation and the original one written in SLI
is supplied in form of raster plots (Fig.~\ref{fig:raster_plot}), 
showing the activity of a subset of the network simulated for 400 ms 
after a transient period of 100 ms. The subset shown consists of the first 
$2.5 \%$ of the neurons of each population created during the initialization of the network. 
Both results have a very similar structure: 
The corresponding subsets show about the same number of events per time. This 
number varies strongly between different populations. Specifically, populations L2/3e and L6e 
fire at much lower frequencies compared to the remaining populations. 
Furthermore, both simulations show some signs of synchrony within populations, indicated by spikes 
of different neurons aligned in vertical lines. This apparent for the populations L2/3e, L2/3i, L4e and 
L5e.
\begin{figure}[tb]
    \centering
    \includegraphics{\figdir raster_plot}
    \caption[Raster plot: PyNEST and SLI]{
        Raster plot showing spontaneous activity of network for 
        (A) the PyNEST implementation and (B) the SLI implementation.
        The simulation and network parameters for both simulations are 
        the same. 
        For each layer, the excitatory population is the upper one shown 
        (total of 1924 neurons) for $400$ ms (with a transient period of 100 ms). 
    }
    \label{fig:raster_plot}
\end{figure}

A more thorough comparison is done on the bases of three statistical quantities: 
Fig.~\ref{fig:spontaneous_activity} shows the population means of single neurons firing rates, 
irregularity as measured by the CV of ISI as well as a measure for synchrony (the Fano factor of the 
PSTH) for both simulations (see Methods, section~\ref{subsec:analysis} for details). 
All parameters were calculated from $20$ repetitions of a simulation for 
$60$ s.
Spikes were recorded from $1000$ neurons of each population, with recording 
starting after a transient period of $0.2$ s. 
For all three quantities, both implementations are in full agreement for the statistical fluctuations
observed. Fluctuations between different initializations differ for each quantities: For the firing rates, 
the largest variance is observed for L5e. For the irregularity, fluctuations are much smaller with the 
highest ones seen in populations L2/3e and L6e. Finally, the variation for synchrony is largest for 
L2/3e and L5e. For synchrony and firing rates, there is a tendency for the fluctuations to be larger for larger
values of the respective quantity -- note however the exception of L5i firing rate, which shows very little
fluctuation.
\begin{figure}[tb]
    \centering
    \includegraphics{\figdir spontaneous_activity}
    \caption[Spontaneous activity: PyNEST and SLI]{
        Measures of spontaneous activity for
        (\textbf{A - C}) the PyNEST implementation and (\textbf{D - E}) the SLI implementation, 
        using the same simulation and network parameters.
        Both implementations are run $20$ times independently (one marker per simulation),
        measuring 1000 spike trains of each population in a simulation for 60 s.
        In case of small 
        fluctuations, individual markers may not be identified due to overlap. 
        \quad (\textbf{A, D}) Population mean of single neuron firing rates.
        \quad (\textbf{B, E}) Irregularity of spike trains measured by the 
        population mean of CV of ISI.
        \quad (\textbf{C, F}) Synchrony of populations quantified 
        by the Fano factor of the PSTH (bin width 3 ms).
    }
    \label{fig:spontaneous_activity}
\end{figure}

In order to get a deeper insight into the dynamics of the simulated network, the single neuron firing 
rates and the CV of ISI of single neurons are examined in Fig.~\ref{fig:single_neuron_activity}.
The observed fluctuations around the population mean are remarkably large and can 
be tracked back to the specific connection rule applied.
Note, however, that despite being large within a population, the 
fluctuations of the population means for different realizations remains small -- an observation 
important for the application of a mean field model. Both aspects will be reviewed in the 
Discussion, section~\ref{sec:discussion}. 

\begin{figure}[tb]
    \centering
    \includegraphics{\figdir single_neuron_activity}
    \caption[Single neuron firing rates and CV of ISI]{
        Firing rates (\textbf{A}) and CV of ISI (\textbf{B}) for single neurons. 
        The data is take from one simulation of the PyNEST implementation, 
        corresponding to a single initialization in Fig.~\ref{fig:spontaneous_activity}
        Statistical fluctuations 
        are indicated by the interquartile ranges (IQR) (boxes extend to 
        the first and third quartile). 
        The median is indicated by a black line, the population mean by a star and 
        whiskers extend to 1.5 IQR (outliers indicated by crosses). 
    }
    \label{fig:single_neuron_activity}
\end{figure}

\FloatBarrier
\section{Mean field theory}
In the following section, The results of the mean field theory are presented 
by showing the results of the prediction of 
population mean of single neuron firing rates and then using these rates 
in order to predict the previously introduced measure of irregularity
as well as the distribution of membrane potentials. 
These results are directly compared to 
the corresponding quantities recorded from the spiking network 
model simulation analyzed in the previous section. 

The firing rates obtained by solving equation 
\eqref{eq:self_consistency_a} are displayed in a bar plot in Fig.%
~\ref{fig:compare_sim_mf_fixed_total_number}. Rates measured in 
simulation and previously shown in Fig.~\ref{fig:spontaneous_activity}
are shown for comparison. As visible, the results of the mean field model 
match those of the simulation to a high degree:
The predicted rates either cover those observed in simulation (in case of population
L4e) or underestimate them to a small degree relative to the measured rate. 
The sequence of populations ordered by increasing firing rates is reproduced.
The comparison can further be quantified: The difference 
$    \Delta \nu_a := \nu_{\text{mf}, a} - \nu_{\text{sim}, a} $
between the mean of simulated rates $\nu_{\text{sim}, a}$ and predicted rates 
$r_{\text{mf}, a}$ for each population $a$ is shown in Table~\ref{tab:diff_fixed_total_number}. 
The mean and standard 
deviation of the absolute values of $\Delta \nu_a$ 
are $(0.31 \pm  0.13)$ Hz. The relative difference
is largest for L2/3e with over $30 \%$, while for the populations 
of layer 4 and 5 as well as for L6e the relative difference is smaller 
than $6 \%$. 
The irregularity measured by the mean CV of ISI for each population is the result of 
plugging in the predicted rates into equation \eqref{eq:CV_ISI_mf}. 
As for the rates, a comparison with simulation data is shown
in Fig.~\ref{fig:compare_sim_mf_fixed_total_number}, 
while numerical results are subsumed in 
Table~\ref{tab:diff_fixed_total_number}, using definitions analogous to those for the rates.
The theoretical results agree well with the measured ones. 
For all populations but L2/3e and L6e, the 
deviation between simulation and mean field theory are of the same order as the fluctuations
between different simulations with deviations of less than $2 \%$ of the measured CV of ISI. 
For L2/3e and L6e, the predicted irregularity is $7 \%$ larger then the values obtained from simulation. 
The theory further predicts the order for sorting the populations according to their
mean CV of ISI. 

% Table of results of comparison
\begin{table}[htb]
    \centering
    \caption[Differences between prediction and simulation]{
        Difference between predicted and simulated population means for 
        firing rates and CV of ISI; absolute and relative to simulated quantities.}
    \label{tab:diff_fixed_total_number}
    \small
    \begin{tabular}{p{2.4cm}| *{8}{x{0.87cm}}} \toprule
    \rowcolor{TableColor}
    \spacedlowsmallcaps{Population}
        & \mc2c{L2/3} & \mc2c{L4} & \mc2c{L5} & \mc2c{L6}  \tn
        \rowcolor{TableColor}
        & \mc1c{e} & \mc1c{i} & \mc1c{e} & \mc1c{i} & \mc1c{e} & \mc1c{i} & \mc1c{e} & \mc1c{i} \tn \hline
        %Population $a$       
        %& L2/3e & L2/3i & L4e & L4i & L5e & L5i & L6e & L6i  \tn[0.2cm]
        $\Delta \nu_a$ / Hz
            & -0.28 & -0.44 &  0.15 & -0.30 & -0.26 & -0.43 & -0.12 & -0.51 \tn[0.2cm]
        $\Delta \nu_a / \nu_{\text{sim}, a}$
            & -0.31 & -0.15 &  0.03 & -0.05 & -0.03 & -0.05 & -0.11 & -0.06 \tn[0.2cm]
        $\Delta \text{CV}_a$
            &   0.063 &   0.016 & -4e-4&   0.002 & 6e-4&   0.002 &   0.067 &   0.007 \tn[0.2cm]
        $\Delta \text{CV}_a / \text{CV}_{\text{sim}, a}$
            &   0.069 &   0.017 & -5e-4&   0.002 & 7e-4&   0.002 &   0.073 &   0.009 \tn[0.2cm]
        \bottomrule
    \end{tabular}
\end{table}

% Comparison mean field / simulation
\begin{figure}[tb]
    \centering
    \includegraphics{\figdir compare_sim_mf_fixed_total_number}
    \caption[Comparing mean field model to simulation]{
        Comparison between mean field theory and spiking network model. 
        Bars indicate the single neuron firing rates predicted by the mean field 
        theory, crosses the calculated ones of 20 simulation (as previously shown in
        Fig.~\ref{fig:spontaneous_activity}). The connection
        rule was set to "fixed total number".
    }
    \label{fig:compare_sim_mf_fixed_total_number}
\end{figure}

Applying equation \eqref{eq:P_V_a} on the predicted rates yields the 
membrane potential distributions shown in 
Fig.~\ref{fig:membrane_potential}. 
The obtained distributions are compared with the normalized histograms of recorded 
membrane potentials of a subpopulation of $n_\text{rec} = 100$ neurons for 
each population. The contribution due to neurons in refractory period is removed
(see Methods, section~\ref{subsec:analysis} for details). 
In contrast to the previous results, the theory and measurements are not in strong agreement. 
For all populations, the predicted distributions are narrower than the recorded ones.
Furthermore, the predicted maximum is shifted towards the threshold (but note that
this does not imply higher firing rates -- in fact, the predicted rates are lower). 
The effect
of neurons coming out of refractory period is underestimated in some cases, 
visible for example both populations of layer 2/3. Here, the mean distributions 
show a step while the kink in the theoretical curves is hardly detectable. 
In the case of population L5i and L6i, however, this behavior is captured well. 
%Similar to the previous results, 
%the agreement is good and the largest deviations are observable for  
%the populations L2/3e and L6e, corresponding to the lowest firing rates and
%largest relative differences as calculated before. A common feature found for all
%populations is that the predicted distributions are narrower than the recorded ones.
%Furthermore, the predicted maximum is shifted towards the threshold (but note that
%this does not imply higher firing rates -- in fact, the predicted rates are lower). 
%The effect
%of neurons coming out of refractory period is underestimated in some cases, 
%visible for example in populations L2/3i and L4i where the mean distributions 
%show a step while the kink in the theoretical curves is hardly detectable. 
%In the case of population L5i and L6i, however, this behavior is captured well. 

% Membrane potentials
\begin{figure}[tb]
    \centering
    \includegraphics{\figdir membrane_potential}
    \caption[Distribution of membrane potentials]{
        Distribution of membrane potentials for each population. 
        Shown are both the results of simulation (histogram) and 
        the predictions of the mean field theory (continuous line). 
        The simulation results are histograms (bins width $\Delta V_\text{m} = 0.25$ mV) 
        of membrane potential recordings 
        of 100 neurons, recorded every 1.0 ms for a simulation time of 1.0 s 
        (after a transitional period of 0.2 s), 
        adjusted for neurons in refractory period (see text). 
        The binning in voltage is the same as applied in Fig.~%
        \ref{fig:single_membrane_potential}. 
        The voltage for neurons in refractory period $V_\text{r} = -65$ mV 
        is indicated by the dashed and dotted line. The threshold is at 
        $\theta = -50$ mV. 
    }
    \label{fig:membrane_potential}
\end{figure}

%\section{Simulation closer to mean field theory }
%In order to assess some of the discrepancies between the predictions of the mean field 
%model and data from simulation, a number of parameters in the simulation are adjusted 
%such that they resemble the ones chosen in the mean field theory more closely. 
%Differentiating parameters in the previous comparison are listed in the 
%Table~\ref{tab:diff_params}.
%\begin{table}[tb]
    %\centering
    %\caption{Parameters chosen differently between simulation and mean field model previously.}
    %\label{tab:diff_params}
    %\begin{tabular}{p{3.0cm} *{2}{|p{5.3cm}}}
        %\rowcolor{TableColor}
        %Parameter & Simulation & Mean field theory   \tn[0.2cm] 
        %Network size &
            %finite; $N_a$ & 
            %none (neglecting finite size effects)
            %\tn[0.1cm] %\hline
        %Connection rule & 
        %"fixed\_total\_number"; binomial distribution of synapse numbers & 
            %"fixed\_indegree"; same synapse number per neuron 
            %\tn[0.1cm] %\hline
        %Synapse type & 
            %current based, exponential shape &
            %voltage based, delta shape
            %\tn[0.1cm] %\hline
        %Synaptic weights & 
            %clipped normal distribution & 
            %Dirac delta distribution 
            %\tn[0.1cm]
    %\end{tabular}
%\end{table}


\FloatBarrier
