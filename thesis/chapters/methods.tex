\section{Methods}
\label{sec:methods}

\subsection{Spiking network simulation}
\label{sub:methods_simulation}
Following Nordlie et al.'s suggestions on 
\emph{good model description practice} \cite{nordlie2009},
the network model is described in prose in the following paragraphs while a detailed 
overview is provided in two tables: The model is summarized in table 
\ref{tab:model_description}, 
whereas specific numerical values of the parameters are displayed in table 
\ref{tab:network_params}. 

\subsubsection{Neuron model}
The neurons are leaky integrate-and-fire neurons with a fixed voltage threshold. 
Below the threshold $\theta$, the dynamics of the membrane potential $V_i(t)$ 
for neuron $i$ are governed by the differential equation 
\begin{equation}
    \tau_\text{m} \,\frac{\text{d} V_i(t)}{\text{d} t} 
            = -(V_i(t) - E_\text{L}) + \frac{\tau_\text{m}}{C_\text{m}} I_i(t) \, .
    \label{eq:leaky_integrator}
\end{equation}
The membrane is specified by its resting potential $E_\text{L}$, 
the time constant $\tau_\text{m}$ and the capacitance $C_\text{m}$.
If at time $t$ the threshold is reached, the neuron emits a spike and remains 
in a refractory period for a fixed time $\tau_\text{rp}$, with the membrane 
potential set to $V_\text{r}$. The total input to the neuron is represented by 
the current $I_i(t)$. 

\subsubsection{Network connectivity}
The network consists of eight cortical populations arranged in four 
layers. Each layer contains an excitatory as well as an inhibitory population. 
A total of 77169 leaky integrate-and-fire neurons are distributed according to the population
sizes given in table \ref{tab:network_params}. The total numbers of excitatory and inhibitory 
neurons are 61843 and 15326, respectively, yielding a ratio of 4.04 of excitatory over inhibitory
neurons. \\
For each combination of pre- and postsynaptic population, pairs of neurons to be connected are drawn
randomly until a fixed number of synapses is reached, allowing for multiple synapses and 
self-connections. Potjans' original model defines the connection probability $P_{\text{conn}, \,ab}$ 
of one neuron in the presynaptic population $a$ to form at least one connection with one neuron in 
the postsynaptic population $b$. For given population sizes $N_a$ and $N_b$, the number of 
synapses $C_{ab}$ is then calculated by
\begin{equation}
    C_{ab} = \frac{\log \left( 1 - P_{\text{conn}, \,ab} \right)}{\log \left( 1 - \frac{1}{N_a N_b} \right)} \, ,
    \label{eq:synapse_numbers}
\end{equation}
the inverse of the formula for connection probabilities derived by Potjans et. al \cite{potjans2014}.

% Model description
\begin{table}[htpb]
    \centering
    \caption{
        Model description according to Nordlie et al. \cite{nordlie2009}. 
        Specific parameters are shown in table \ref{tab:network_params}.
        }
    \label{tab:model_description}
    \begin{tabular}{m{3.1cm} p{10cm}}
        \rowcolor{TableColor}\multicolumn{2}{l}{Model summary} \\
        Populations     &   8 cortical populations\\
        Topology        &   --\\
        Connectivity    &   Random connections with fixed number of synapses for 
                            each combination of pre- and postsynaptic population\\
        Neuron model    &   Leaky integrate-and-fire, fixed voltage threshold, fixed 
                            absolute refractory period\\
        Synapse model   &   exponential-shaped postsynaptic current\\
        Plasticity      &   --\\
        Input           &   Independent fixed-rate Poisson spike trains\\
        Measurements    &   Spike activity, membrane potentials \tnn

        \rowcolor{TableColor} Populations & \\
        Layers          &   L2/3, L4, L5, L6 \\
        Cortical network&   one excitatory (e) and one inhibitory (i) population per layer\\
        Size            &   population specific size 
                            (see table \ref{tab:network_params}) \tnn

        \rowcolor{TableColor} Connectivity & \\
        Type            &   Random connectivity with independently chosen pre- and postsynaptic
                            neurons; fixed total number of connections between two populations
                            (see table \ref{tab:network_params}) \\
        Weights         &   Fixed; drawn from clipped Gaussian distributions 
                            ($w > 0$ for excitatory, $w~<~0$ for inhibitory)\\
        Delays          &   Fixed; drawn from clipped Gaussian distributions ($d~>~0$);
                            multiples of computation step size \tnn

        \rowcolor{TableColor}\multicolumn{2}{l}{ Neuron and synapse model} \\
        Name            &   iaf neuron\\
        Type            &   Leaky integrate-and-fire, exponential-shaped current inputs\\
        Subthreshold \newline dynamics of \newline neuron~$i$
                        &   {$\!\begin{aligned} 
                            \tau_\text{m} \,\frac{\text{d} V_i(t)}{\text{d} t} 
                                    &= -(V_i(t) - E_\text{L}) + \frac{\tau_\text{m}}{C_\text{m}} I_i(t)
                                        &\text{if}\quad& t > t^* + \tau_\text{rp} \\ 
                                V_i(t)        &= V_\text{r}  &\text{else}& \\[0.2cm]
                                I_i(t) &= C_\text{m} \sum_j J_{ij} \sum_k \delta (t - t_j^k - d_{ij})  
                            \end{aligned}$}  \\
                            &THIS IS NOT TRUE -- STATE THE ACTUAL ODE! \\
        Spiking         &   If $\,\,V_i(t_-) < \theta \quad \land \quad V_i(t_+) \ge \theta$: \\
                        &   \quad 1. set $t^* = t$    \\
                        &   \quad 2. emit spike with time stamp $t^*$ \tnn

        \rowcolor{TableColor} Input & \\
        Type            &   Independent Poisson spikes to iaf neurons
                            (see table \ref{tab:network_params})
    \end{tabular}
\end{table}

% Network parameters
\begin{table}[htpb]
    \centering
    \caption{
        Network parameters
        }
    \label{tab:network_params}
    \begin{tabular}{p{3.5cm} *{8}{x{1.2cm}}}
        \rowcolor{TableColor}\multicolumn{9}{l}{Populations and inputs} \tn
        Name        
            & L23e & L23i & L4e & L4i & L5e & L5i & L6e & L6i  \tn
        Population size, $N$   
            & 20683 & 5834 & 21915 & 5479 & 4850 & 1065 & 14395 & 2948 \tn
        External inputs, $k_\text{ext}$ 
            & 1600 & 1500 & 2100 & 1900 & 2000 & 1900 & 2900 & 2100 \tn
        Background rate     
        & 8 Hz \tnn

        \rowcolor{TableColor}\multicolumn{9}{l}{Connection probabilities between pre- and postsynaptic populations} \tn
        \backslashbox{post}{pre}
            & L23e & L23i & L4e & L4i & L5e & L5i & L6e & L6i  \tn
        L23e
            & 0.101 & 0.169 & 0.044 & 0.082 & 0.032 & 0.000 & 0.008 & 0.000 \tn 
        L23i
            & 0.135 & 0.137 & 0.032 & 0.051 & 0.075 & 0.000 & 0.004 & 0.000 \tn 
        L4e
            & 0.008 & 0.006 & 0.050 & 0.135 & 0.007 & 0.000 & 0.045 & 0.000 \tn 
        L4i
            & 0.069 & 0.003 & 0.079 & 0.160 & 0.003 & 0.000 & 0.106 & 0.000 \tn 
        L5e
            & 0.100 & 0.062 & 0.051 & 0.006 & 0.083 & 0.373 & 0.020 & 0.000 \tn 
        L5i
            & 0.055 & 0.027 & 0.026 & 0.002 & 0.060 & 0.316 & 0.009 & 0.000 \tn 
        L6e
            & 0.016 & 0.007 & 0.021 & 0.017 & 0.057 & 0.020 & 0.040 & 0.225 \tn 
        L6i
            & 0.036 & 0.001 & 0.003 & 0.001 & 0.028 & 0.008 & 0.066 & 0.144 \tnn

        \rowcolor{TableColor}\multicolumn{9}{l}{Further connectivity} \tn
        $J \pm \delta J$    
            &  \multicolumn{3}{l}{$87.8 \pm 8.8 \,\text{pA}$}
            &  \multicolumn{5}{l}{Excitatory synaptic strengths} \tn
        $g$    
            &  \multicolumn{3}{l}{$-4$}
            &  \multicolumn{5}{l}{Relative inhibitory synapse strength} \tn
        $d_e \pm \delta d_e$    
            &  \multicolumn{3}{l}{$1.5 \pm 0.75 \,\text{ms}$}
            &  \multicolumn{5}{l}{Excitatory synaptic transmission delays} \tn
        $d_i \pm \delta d_i$    
            &  \multicolumn{3}{l}{$0.8 \pm 0.4 \,\text{ms}$}
            &  \multicolumn{5}{l}{Inhibitory synaptic transmission delays} \tnn

        \rowcolor{TableColor}\multicolumn{9}{l}{Neuron model} \tn
        $\tau_\text{m}$    
            &  \multicolumn{3}{l}{$10 \,\text{ms}$}
            &  \multicolumn{5}{l}{Membrane time constant} \tn
        $\tau_\text{ref}$    
            &  \multicolumn{3}{l}{$\hphantom{0}2 \,\text{ms}$}
            &  \multicolumn{5}{l}{Absolute refractory period} \tn
        $\tau_\text{syn}$    
        &  \multicolumn{3}{l}{$\hphantom{0}0.5 \,\text{ms}$}
            &  \multicolumn{5}{l}{Postsynaptic current time constant} \tn
        $C_\text{m}$    
            &  \multicolumn{3}{l}{$250 \,\text{pF}$}
            &  \multicolumn{5}{l}{Membrane capacity} \tn
        $E_\text{L}$    
            &  \multicolumn{3}{l}{$-65 \,\text{mV}$}
            &  \multicolumn{5}{l}{Leaky rest potential} \tn
        $V_\text{reset}$    
            &  \multicolumn{3}{l}{$-65 \,\text{mV}$}
            &  \multicolumn{5}{l}{Reset potential} \tn
        $\theta$    
            &  \multicolumn{3}{l}{$-50 \,\text{mV}$}
            &  \multicolumn{5}{l}{Fixed firing threshold} \tn
    \end{tabular}
\end{table}

\subsubsection{Synaptic input}
The spikes of neurons are modeled as delta functions. Accordingly, the 
input current of neuron $i$ can be described as the sum over arriving 
spike trains, 
\begin{equation}
    I_i(t) = C_\text{m} \sum_j J_{ij} \sum_k \delta (t - t_j^k - d_{ij}) \, ,
    \label{eq:input_current}
\end{equation}
where $t_j^k$ is the time the $k$-th spike by neuron $j$ was emitted and the 
delay between neuron $i$ and $j$ is set to $d_{ij}$. 


Explain synapse dynamics, the change from PSP to PSC, etc.


\subsection{Mean field model}
The derivation of a mean field theory for the layered network goes along the 
lines of the work by Brunel \cite{brunel2000}.
It starts of with a simplified model of 
$N$ neurons. Each neuron receives input from the network by $C$ synapses, $C_E$ and $C_I$ of 
which connect to excitatory and inhibitory neurons, respectively. 
Furthermore, each neurons receives $C_\text{ext} = C_E$ connections from 
external excitatory neurons.
The synapse numbers 
are determined by the relative size of the two populations through the factor
\begin{equation}
    \epsilon = \frac{C_E}{N_E} = \frac{C_I}{N_I} \,.
    \label{eq:epsilon}
\end{equation}
A central assumption is the sparsity of the network, expressed by $\epsilon \ll 1$.
Guided by anatomical estimates for the neocortex, the population sizes are set to
$N_e = 0.8N$ excitatory and $N_i = 0.2N$ inhibitory neurons. Taking \eqref{eq:epsilon}
this implies 
\begin{equation}
    C_I = \gamma C_E 	
 \label{eq:C_I}
\end{equation}
with $\gamma = 0.25$. The synaptic weights in this model are set to $J$ for 
excitatory presynaptic neurons and to $-g\, J$ for inhibitory ones. 
The delays are fixed uniformly to $d$ for all synapses. 

At the heart of the mean field model
is the transition from the deterministic description of membrane potential 
dynamics according to \eqref{eq:leaky_integrator} and \eqref{eq:input_current} to
a probabilistic formulation. Here, the input is modeled as a time-varying average part
$\mu(t)$ plus a fluctuating gaussian part with amplitude $\sigma(t)$:
\begin{equation}
    I_i(t) = C_\text{m} \tau_\text{m} \left[ \mu(t) + \sigma(t)	\sqrt{\tau_\text{m}} \eta_i(t) \right] \, .
    \label{eq:input_random}
\end{equation}
The random fluctuations are described by gaussian white noise $\eta_i(t)$ with 
$\langle  \eta_i(t)\rangle = 0$. The model explicitly excludes correlations, 
both in time and between different neurons, i.e.
\begin{equation}
    \langle \eta_i(t) \: \eta_j(t') \rangle = \delta_{ij} \: \delta(t - t')	\, . 
    \label{eq:no_correlations}
\end{equation}
The latter assumptions is by no means trivial and has to be tested for the spiking model
that is to be described with this mean field approach. 

The average input $\mu(t)$ and amplitude of fluctuations $\sigma(t)$ are linked to the 
average firing rate $\nu(t)$ by the equations
\begin{equation}
    \begin{split}
        \mu(t)          &= \mu_l(t) + \mu_\text{ext} \\
        \text{with} \qquad \mu_l(t)        &= C_E \, J (1 - \gamma g) \nu(t - d) \tau_\text{m} \\
        \mu_\text{ext}  &= C_E J \nu_\text{ext} \tau_\text{m}
        \label{eq:mu}
    \end{split}
\end{equation}
\begin{equation}
    \begin{split}
        \sigma^2(t)     &= {\sigma_l}^2(t) + {\sigma_\text{ext}}^2 \\
        \text{with} \qquad {\sigma_l}^2(t)       
                        &= C_E \, J^2 (1 + \gamma g^2) \nu(t - d) \tau_\text{m} \\
        {\sigma_\text{ext}}^2  &= C_E J^2 \nu_\text{ext} \tau_\text{m}
        \label{eq:sigma}
    \end{split}
\end{equation}

THIS IS EXTENDED TO:

\begin{align}
    \mu_a        &= 
        \tau \sum_{b \in \text{pop.}} C_{ab} \, J_{ab} \, \nu_b 
        + \tau C_{a, ext} \, J_{a, ext} \, \nu_{ext}
\intertext{and fluctuation}
    {\sigma_a}^2 &= 
        \tau \sum_{b \in \text{pop.}} C_{ab} \, {J_{ab}}^2  \, \nu_b
        +
        \tau C_{a, ext} \,{J_{a, ext}}^2 \,\nu_{ext}
\end{align}
