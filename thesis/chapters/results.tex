\section{Results}
\label{sec:results}

\subsection{Spiking network model}
Comparison of spontaneous activity for Potjans' original model and 
the pyNEST implementation. 

Explain calculation of derived parameters (rates, CV of ISI, synchrony).



\begin{figure}[htpb]
    \centering
    \includegraphics[width=0.8\linewidth]{../figures/spon_activity}
    \caption{Spontaneous activity of network in the pyNEST implementation. 
        Recordings $10\%$ of the neurons of all populations are displayed.
        Firing rate, CV of interspike intervals and synchrony are calculated 
        based on a simulation of 20 seconds of the full scale network. For 
        each layer, the excitatory population is the upper one shown 
        (with lighter color).
    }
    \label{fig:spon_activity}
\end{figure}
\begin{figure}[htpb]
    \centering
    \includegraphics[width=0.8\linewidth]{../figures/spon_activity_sli}
    \caption{Spontaneous activity of network of the original implementation (written in SLI). 
        All parameters and analysis as in Figure \ref{fig:spon_activity}.
    }
    \label{fig:spon_activity_sli}
\end{figure}

Analyze statistical fluctuations of the derived parameters(rates, etc.) 
-> box plots

\subsection{Mean-field theory}

\subsection{Stability of the network}
