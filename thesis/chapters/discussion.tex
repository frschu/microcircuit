\section{Discussion}
\label{sec:discussion}

% Spiking network
The reimplementation of Potjans and Diesmann's spiking network simulation  
confirmed the expectation: The original model's results where reproduced 
well within the statistical fluctuations. Fluctuations of measures on 
the population level between different instantiations of the model are 
relatively small, indicating low dependency on the actual wiring. 
This characteristic can also be interpreted as an indication that a mean 
field approach is suited well for describing the corresponding network:
First order statistical parameters are preserved within different instantiations, 
those of higher order are subject to larger fluctuations.
\emph{(is this even true?)} However, this assertion has to be restricted to 
random networks as more detailed structure, for example induce by learning, 
can lead to stronger effects of correlation (see for example \cite{staude2010higher}).
The fluctuations along neurons within one population and a single instantiation
of the network were shown to be \emph{quite large} both for the rate and the 
CV of ISI. This can be tracked back to the large deviations of input between 
different neurons: Due to randomly choosing the synapses and adding further 
variability by distributing the synapse number (as shown for model validation), 
some neurons will receive much more excitatory or inhibitory input than others 
and fire accordingly. 

% Mean field theory
Introducing the mean field theory for the spiking network simulated before 
turned out to be surprisingly successful. 
The formal extension 
from the original model of two populations has been a rather small step, while
the numerical implementation revealed a number of obstacles. The resulting
algorithm, nonetheless, is a convenient and computationally cheap tool for 
predictions. It has been shown to predict central quantities 
of the network activity to a high degree of accuracy.  

% Rates
The predicted population means of single neuron firing rates differ by 
$\sim 0.5$ Hz when compared to simulation data. For the populations firing at 
higher rates, i.~e. those of layers 4 and 5 as well as L6i, this corresponds 
to $< 10 \%$ of the respective rates. Yet, for the much more quiescent populations 
L2/3e and L6e, the relative disagreement is much more drastic ($\sim 50 \%$). 
For all populations but L6e, the predicted rates are lower than the measured ones.  
One reason for this underestimation could be the negligence of correlation: 
Larger correlations among excitatory input can lead to higher fluctuations 
and thus higher spiking rates~\cite{staude2010higher} \emph{<-???}. This 
assertion has to be taken with care, though, as correlation in inhibitory 
populations can cancel the effect \emph{citation!! (I just remember Sadra saying that :) )}. 

% CV of ISI
The prognosis for the irregularity of spike trains, measured by the coefficient of
variation of interspike intervals (CV of ISI) shows a similar picture:
Excluding the abiding populations L2/3e and L6e, the deviation is $\sim 5 \%$
and the theoretical value is always larger than the measured one. The two remaining 
populations are predicted to have the highest irregularity (not observable in simulation), 
and the deviation to measured values is about $15 \%$. 
One explanation for the reported differences can be an estimation bias of the CV of ISI 
using spike trains of finite lengths arising as a significant ratio of the neurons in the 
population is either not included at all ($n_i < 2$) or the part of the distribution 
covering higher CV of ISI is not represented well (see \cite{nawrot2010analysis} for details). 
This is especially critical for the populations with low rates (L2/3e, L6e).
What cannot be accounted for by this bias would indicate a lower irregularity than that 
of the respective stationary Gaussian process, pointing towards temporal correlations
for example introduced by the synapse model 
(see e.~g. \cite{brunel1999fast} for the case of $\alpha$ synapses in a simpler context). 

% Membrane potentials
The third and last measure predicted by the mean field model, the distribution of 
membrane potentials, agrees well with the obtained simulation data, too. 
Both the shape and position of the distribution are reproduced despite being slightly 
narrower than the measured histograms. The kink at $V_\text{r}$ due to neurons exiting the 
refractory period is reproduced but smaller than the measured one, indicating that the diffusion
from this point is slower than assumed. The largest deviations are again observed for 
the populations L2/3e and L6e, where the measured distributions are shifted towards 
the resting potentials. For L4i, this is counterintuitive as the predicted rate is 
lower than the measured one and would again point towards the impact of correlations:
Even if the membrane potential spends most time close to the resting potential, 
a large number of excitatory spikes arriving in a short time would lead to a quick rise and
firing without deforming the distribution significantly.  

% Summary of results, explanations and shortcomings
In summary, the analytical ansatz holds what has been hypothesized: 
The first order measures of the simulated spiking network model of the neocortex 
can be predicted by a mean field theory assuming uncorrelated Gaussian input.
Cuts have to be made in regimes where correlations play a significant role. 
In the present neocortical model, temporal correlations 
cause by the current based synapse type (postsynaptic currents are induced with 
exponential decay in time, not instantly as assumed in the mean field model)
\emph{are shown to have a significant effect (by simulation with delta synapses...)}. 
Furthermore, correlations between neurons, i.~e. synchrony, induces differences in 
rates. Presumably, this is most relevant in populations firing at very low rates ($< 1$ Hz), 
leading to the observed deviations between theory and simulation for populations 
L2/3e and L6e. Yet another possible factor for this disparity could be that the
mean field theory does not capture well the effect of inhomogeneity introduced 
by changing the weight of the connection L4e to L2/3e. 
A fraction of the stated quantitative disagreements could be solved by 
including further sources of variance into the model.
The distribution of $J$ could by included into the variance directly as
in \cite{amit1997model}. The variance could be corrected for the synapse 
model like Sadra did \cite{sadeh2014mean}. Temporal correlations induced
by the synapse model as well as effects due to inhomogeneity in synapse 
numbers are covered by \cite{brunel1999fast}. 
\emph{As simulations show, the agreement improves / does not improve...}

% Use of the theory, limits
The value of the analytical framework developed is twofold: On the one hand, 
it provides a useful tool for predicting the activity of a spiking network model
over a large range of parameters \emph{as observed for external frequency $\nu_{ext}$ 
and relative inhibitory synapse strength $g$}. On the other hand, it represents an
essential means for identifying relevant measures and understanding the emergent 
dynamics of the complex systems under consideration. This is a hard and arguably the 
most important task this realm of computational neuroscience is aimed at. The long lasting
debate over whether neural coding is rate based or exploits precise timing and correlations
may at some point be solved excluding one or the other option using a sensible framework
of biological data, spiking network simulation and analytical arguments. 
On a less ambitious scale, the presented models can at least indicate by how much 
correlations effect the observed rates. 

% Outlook
Besides the proposed refinements on a fine scale
the presented combination of the spiking network model and the mean field approach
could serve as a framework to tackle further questions. 
The implementation in pyNEST works as a convenient basis for extensions 
such as the inclusion of newly available experimental data.
Furthermore, different neuron types, 
especially concerning different interneuron populations, may be included, 
making the simulation a viable means for testing hypothesis about their 
role. 
Another possible extension already introduced in the 
original model \cite{potjans2014} is unspecific input from a thalamic population. 
When applying this input for for short bursts (e.~g. 10 ms), 
this can by used in order to examine a possibility for propagation of information
along the different layers in time and thus assessing the 
path of signal processing within the neocortex
(\emph{see blabla for an overview of 
the presumed way of information processing over the layers}). 
The mean field approach, however, would have to be extended to a non-equilibrium 
regime since it lacks temporal resolution at the present state. 
When focusing on neural computation, one might also include specific input.
An interesting context is orientation selectivity, using 
oriented input together with equipping neurons with tuning curves.
To this end, the mean field approach can be extended to single neurons 
as shown for example by Sadeh~\cite{sadeh2015orientation} 
in the case of one excitatory and one inhibitory population.




