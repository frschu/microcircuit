\section{Discussion}
\label{sec:discussion}

% Spiking network
The reimplementation of Potjans and Diesmann's spiking network simulation  
confirmed the expectation: The original model's results where reproduced 
well within the statistical fluctuations. The fluctuations of the calculated population 
means between different instantiations of the model are 
relatively small, a characteristic that can be interpreted as an indication that a mean 
field approach is suited well for describing the corresponding network:
The parameters describing the main features of the network activity
depend strongly on average input and connection numbers,
and much less on the actual wiring.
However, this assertion has to be restricted to 
random networks as more detailed structure, for example induce by learning, 
can lead to stronger effects of correlation (see for example \cite{staude2010higher}).
The fluctuations along neurons within one population and a single instantiation
of the network were shown to be large both for the firing rate and the 
CV of ISI. This can be tracked back to the large deviations of input between 
different neurons: Due to randomly choosing the synapses and adding further 
variability by distributing the synapse number (as shown for model validation, 
Methods, section \ref{subsec:methods_simulation}), 
some neurons will receive more excitatory or inhibitory input than others 
and fire accordingly. 
The measurement of the synchrony indicates that there is a significant 
amount of correlation among spike trains of different neurons. This has to be taken 
into account when interpreting the results of the mean field model. 

% Mean field theory
Introducing the mean field theory for the simulated spiking network 
turned out to be remarkably successful. 
The formal extension 
from the original model of two populations has been a rather small step, while
the numerical implementation revealed a number of obstacles. The resulting
algorithm, nonetheless, is a convenient and computationally inexpensive tool for 
predictions. It has been shown to predict central quantities 
of the network activity to a high degree of accuracy.  

% MF: Rates
The predicted population means of single neuron firing rates differ by 
$\sim 0.3$ Hz when compared to simulation data. For the populations firing at 
higher rates, i.~e. those of layers 4 and 5 as well as L6i, this corresponds 
to $<6 \%$ of the respective rates. Yet, for the more quiescent population L2/3e, 
the relative disagreement is much more drastic ($\sim 30 \%$). 
For all populations but L4e, the predicted rates are lower than the measured ones.  
One reason for this underestimation could be the negligence of correlation: 
Larger correlations among excitatory input can lead to higher fluctuations 
and thus higher spiking rates~\cite{staude2010higher}. This 
assertion has to be taken with care, though, as correlation in inhibitory 
populations can cancel the effect \emph{(citation!! Sadra emphasized that)}. 

% MF: CV of ISI
The prognosis for the irregularity of spike trains, measured by the coefficient of
variation of interspike intervals (CV of ISI) turns out to be more exact.:
Excluding the populations L2/3e and L6e, the deviation between theory and simulation 
is $<2 \%$. The two remaining 
populations are predicted to have the highest irregularity (as observed in simulation), 
and the deviation to measured values is about $7 \%$. 
One explanation for the remaining differences can be an estimation bias of the CV of ISI 
using spike trains of finite lengths arising as a significant ratio of the neurons in the 
population is either not included at all (if the number of spikes is $ < 2$) or the part of the distribution 
covering higher CV of ISI is not represented well (see \cite{nawrot2010analysis} for details). 
This is especially critical for the populations with low rates (L2/3e, L6e).
What cannot be accounted for by this bias would indicate a lower irregularity than that 
of the respective stationary Gaussian process, pointing towards temporal correlations
introduced for example by the synapse model 
(see e.~g. \cite{brunel1999fast} for the case of $\alpha$ synapses in a simpler context). 

% MF: Membrane potentials
The third and last measure predicted by the mean field model, the distribution of 
membrane potentials, shows more visible disagreement with the obtained simulation 
data than the previous ones. Although the general shape of the distributions is reproduced, the predicted
ones are narrower than the measured histograms. The maximum is further shifted 
towards the threshold. 
The kink at the resting potential $V_\text{r}$ due to neurons exiting the 
refractory period is reproduced but less pronounced than the measured one, indicating that the diffusion
away from this point is slower than assumed. The largest deviations are observed for 
the populations L2/3e and L6e, where the measured membrane potentials have a considerably 
high density below $V_\text{r}$.
Again, the fact that predicted rates are lower than the measured ones 
despite the distributions' means being closer to the threshold
would again point towards the impact of correlations:
Even if the membrane potential spends most time close to the resting potential, 
a large number of excitatory spikes arriving in a short time would lead to a quick rise and
firing without affecting the distribution significantly.  
This interpretation is in agreement with the measured degree of synchrony. 
Ultimately, the wide distribution of single neuron firing rates within each population
can be expected to widen the distribution. 

%The third and last measure predicted by the mean field model, the distribution of 
%membrane potentials, agrees well with the obtained simulation data, too. 
%Both the shape and position of the distribution are reproduced despite being slightly 
%narrower than the measured histograms. The kink at $V_\text{r}$ due to neurons exiting the 
%refractory period is reproduced but smaller than the measured one, indicating that the diffusion
%from this point is slower than assumed. The largest deviations are again observed for 
%the populations L2/3e and L6e, where the measured distributions are shifted towards 
%the resting potentials. For L4i, this is counterintuitive as the predicted rate is 
%lower than the measured one and would again point towards the impact of correlations:
%Even if the membrane potential spends most time close to the resting potential, 
%a large number of excitatory spikes arriving in a short time would lead to a quick rise and
%firing without deforming the distribution significantly.  

% Summary of results, explanations and shortcomings
In summary, the hypothesis for the analytical ansatz is confirmed: 
The considered activity measures of the simulated spiking network model of the neocortex 
can be predicted by a mean field theory assuming uncorrelated Gaussian input.
Possible explanations for the remaining deviations are (i) correlations between neurons, 
yielding an input to single neurons different from the assumed white Gaussian noise;
(ii) temporal correlations induced by the synapse type different from the 
delta synapses of the mean field model; and finally (iii) fluctuations in the 
input of single neurons. 
%\emph{Comparison with simulations where the according parameters are adapted 
    %have show the individual effects}.

% Use of the theory, limits
The value of the analytical framework developed is twofold: On the one hand, 
it provides a useful tool for predicting the activity of a spiking network model
over a large range of parameters 
%\emph{as observed for external frequency $\nu_{ext}$ 
%and relative inhibitory synapse strength $g$}. 
On the other hand, it represents an
essential means for identifying relevant measures and understanding the emergent 
dynamics of the complex systems under consideration. This is a hard and one of the 
most important tasks in this field. 
The long lasting
debate over whether neural coding is rate based or exploits precise timing and correlations
may at some point be solved by
excluding one or the other option using a sensible framework
of biological data, spiking network simulation and analytical arguments. 
On a less ambitious scale, the presented models can at least indicate by how much 
correlations effect the observed rates. 

% Outlook
The presented combination of the spiking network model and the mean field approach
could serve as a framework to tackle further questions. 
The implementation in PyNEST could work as a convenient basis for extensions 
such as the inclusion of newly available experimental data.
Furthermore, different neuron populations, 
especially concerning different interneuron classes, may be included, 
making the simulation a viable means for testing hypothesis about their 
role. 
Another possible extension already introduced in the 
original model \cite{potjans2014} is unspecific input from a thalamic population. 
When applying this input for for short bursts (e.~g. 10 ms), 
this can by used in order to examine a possibility for propagation of information
along the different layers in time and thus assessing the 
path of signal processing within the neocortex.
(\emph{see ??? for an overview of 
the presumed way of information processing over the layers})
The mean field approach, however, would have to be extended to a non-equilibrium 
regime since it lacks temporal resolution at the present state. 
When focusing on neural computation, one might also include specific input.
An interesting context is orientation selectivity, using 
oriented input resulting in neuronal tuning curves.
To this end, the mean field approach can be extended to single neurons 
as shown for example by Sadeh~\cite{sadeh2015orientation} 
for the case of one excitatory and one inhibitory population.




