\section{Abstract}
\label{abstract}
In the search for understanding the basic functions of the neocortex, 
linking the anatomical structure to population dynamics has been a mayor
focus of research. Experimental data in form of local connectivity data
and cell-type specific activity is increasing at a fast past but remains
mainly inconclusive. Simultaneously, spiking network simulations using 
leaky integrate-and-fire neurons 
have been successfully developed and used as 
explanatory schemes. 
A sensible analytical framework is under construction although a 
large number of open questions persist. 
A full-scale spiking network model of the local cortical microcircuit
was established by Potjans and Diesmann~\cite{potjans2014} in 2014, 
using a large number of the experimental data available and reproducing
some of the main features of spiking activity. 
On the analytical side, a rate based mean field theory for spiking neuron
networks has been developed by Brunel~\cite{brunel2000} 
and widely applied since then. 
Here, the spiking network model of the microcircuit is reimplemented, 
compared with the original one and further analyzed.
Aiming for a more thorough understanding as well as a computationally less
expensive tool, the existing mean field theory is extended to the given network. 
The predictions are tested against simulated data, namely predicting 
firing rates, membrane potential distributions as well as a measure 
for the irregularity of spikes (CV of ISI). \emph{Finally, in order to test 
the dependence on a number of parameters, a simulation adapted to the 
mean field theory is contrasted with the previous results.}
